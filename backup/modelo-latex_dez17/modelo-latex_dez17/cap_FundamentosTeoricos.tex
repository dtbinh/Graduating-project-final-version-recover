%TCIDATA{LaTeXparent=0,0,relatorio.tex}
                      
\chapter{Fundamentos Te�ricos}\label{CapFundamentosTeoricos}

% Resumo opcional. Comentar se n�o usar.
%\resumodocapitulo{Resumo opcional.}

\section{Introdu��o}



\section{Controladores l�gicos program�veis}

	Os controladores l�gicos program�veis (CLPs) s�o aparelhos que operam uma l�gica computacional e controlam m�quinas e processos industriais a partir de suas entradas e sa�das digitais e anal�gicas \cite{article:nema:2005}. O CLP centraliza as informa��es de um processo ou parte dele, sendo capaz criar diferentes malhas de controle dentro de sua programa��o e realizar uma automa��o industrial completa. 
		
	De modo a entender como um CLP se encaixa em uma malha de controle de processo, cabe saber como � feita a leitura de uma entrada anal�gica e digital. Em ambas ocorre uma convers�o anal�gico para digital (A/D). No caso da entrada digital, esta pode receber apenas dois n�veis de tens�o, um alto e outro baixo. A convers�o do sinal � feita para um bit de mem�ria do CLP. A Tabela~\ref{tab:clp_DIs} apresenta duas l�gicas de uma porta digital de um CLP.
	\begin{table}[ht]
	\centering
	\caption{L�gica da convers�o A/D de uma porta digital}
	\label{tab:clp_DIs}
	\begin{tabular}{c|c|c|}
	\cline{2-3}
	\textbf{}                                       & \multicolumn{2}{c|}{Convers�o para um BIT}       \\ \hline
	\multicolumn{1}{|c|}{\textbf{Escala de tens�o}} & \textbf{L�gica direta} & \textbf{L�gica reversa} \\ \hline
	\multicolumn{1}{|c|}{ALTO (e.g. 24V)}           & 1                      & 0                       \\ \hline
	\multicolumn{1}{|c|}{BAIXO (e.g. 0V)}           & 0                      & 1                       \\ \hline
	\end{tabular}
	\end{table}
	
Definindo
\begin{equation} \label{eq:k}
\phi \triangleq \dfrac{a}{1+b^2}
\end{equation}

Em \eqref{eq:k} mostra-se que