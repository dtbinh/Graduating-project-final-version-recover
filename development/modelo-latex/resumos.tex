%TCIDATA{LaTeXparent=0,0,relatorio.tex}

\resumo{Resumo}{Este texto apresenta a proposta de um novo teorema para se determinação da estabilidade local da classe específica de sistemas não lineares na forma $\dot{x} = f(x)$ com pontos de equilíbrio na origem, modelados como sistemas fuzzy Takagi-Sugeno via não linearidade de setor local, por meio da Teoria de Estabilidade de Lyapunov e utilizando-se Desigualdades Matriciais Lineares - LMIs - e do Teorema de Finsler. Este novo teorema proposto também permite obter a estimativa do domínio de atração do ponto de equilíbrio na origem, quando este for estável, utilizando-se o conceito de superfície de Lyapunov. Para sistemas em que a origem não é um ponto de equilíbrio, utiliza-se o artifício de mudança de variável, para transportar o ponto de equilíbrio para a origem sem que se perca a generalidade das respostas do sistema. Neste trabalho é mostrado que o teorema aqui proposto produz resultados menos conservadores que outros teoremas de análise de estabilidade local encontrados na literatura.}

\vspace*{2cm}

\resumo{Abstract}{This paper proposes a new theorem for determining the local stability of the specific class of nonlinear systems in the form $ \ dot {x} = f (x) $ with equilibrium points at the origin, modeled as Takagi-Sugeno fuzzy systems via nonlinearity of local sector, through the Lyapunov Stability Theory and using Linear Matrix Inequalities - LMIs - and Finsler's Theorem. This new proposed theorem also allows to obtain the estimation of the attraction domain of the equilibrium point at the origin, when it is stable, using the Lyapunov surface concept. For systems where the origin is not a point of equilibrium, the variable-shift artifice is used to carry the equilibrium point to the origin without losing the generality of the system responses. In this work it is shown that the theorem proposed here produces less conservative results than other local stability analysis theorems found in the literature.}