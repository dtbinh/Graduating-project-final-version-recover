%TCIDATA{LaTeXparent=0,0,relatorio.tex}

\chapter{\textit{Plot} Retrato de Fase\label{AnEsquematicos}

x1 = x1\_min:x1\_step:x1\_max;

x2 = x2\_min:x2\_step:x2\_max;

[X1,X2] = meshgrid(x1,x2);

F1= f1(X1, X2);

F2= f2(X2, X2);

hold off

figure;clf

streamslice(X1,X2,F1,F2,2); \% https://www.mathworks.com/help/matlab/ref/streamslice.html

xlabel('x\_1');ylabel('x\_2');

line([0],[0],'marker','o','linestyle','none','markerfacecolor','r') \%plots origin as equilibrium point

hold off;

\chapter{Plot Condi��o de Inclus�o (Estimativa de regi�o de atra��o) para P constante\label{AnEsquematicos

function level\_curve(P, gamma, color)

\%Plots the level curve V(x) = x'*P*x = gamma, given P and gamma.

for i = 1:length(P)


if length(P) > 1



Pi = P{i{1;
else
Pi = P{i;
end
%a = Pi(1,1);
b = Pi(1,2); % = Pi(2,1);
c = Pi(2,2);
dP = det(Pi);

%[V,D] = eig(Pi);
%v1=V(:,1);
%v2=V(:,2);
%figure;plot(v1(1),v1(2),'r*',v2(1),v2(2),'b*');
%hold on

%Calculo
x1max=sqrt(c*gamma/dP);
steps = (2*x1max)/1e4;
x1=-x1max:steps:x1max;
x2a=($-b*x1+sqrt(c*gamma-x1.^2*dP))/c$;
x2b=($-b*x1-sqrt(c*gamma-x1.^2*dP))/c$;

%Plot
hold on
plot(x1,x2a,color)
plot(x1,x2b,color)
hold off
grid
volume = det(inv(Pi)*gamma); 
end
end
